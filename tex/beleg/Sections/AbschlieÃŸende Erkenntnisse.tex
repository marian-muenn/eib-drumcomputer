\section{Herausforderungen im Projektverlauf}
Während des Projekts traten verschiedene Herausforderungen auf, die im Folgenden zusammenge-
fasst werden:
\begin{description}
\item[Einschränkungen bei der Schaltungssimulation:] Die Erzeugung von Rauschen in der Schal-
tung erfolgt durch die unübliche Nutzung eines Transistors. Dies führte dazu, dass eine
vollständige Simulation der Schaltung nicht möglich war, um deren ordnungsgemäße Funk-
tionsweise zu überprüfen.
\item[Hastige Implementierung eines einstellbaren Tiefpassfilters:] Die Einführung eines einstell-
baren Tiefpassfilters kurz vor Projektabschluss führte zu einer unzureichenden  Überprüfung
der Schaltung, was zu Fehlern führte.
\item[Unzureichende Qualitätskontrolle bei der Zusammenarbeit:] Da jedem Projektmitglied spezifische Aufgabenbereiche zugewiesen wurden, erfolgte eine unzureichende Überprüfung der
Arbeit anderer Teammitglieder. Ein Beispiel hierfür ist die Verwendung eines falschen Foot-
prints für den Transistor, was zu erheblichem Zeitverlust bei der Fehlersuche führte.
\item[Fehleinschätzung der Projektkosten]: Die geschätzten Kosten für das gesamte Projekt wur-
den erheblich unterschätzt, unter anderem aufgrund der ungeprüften Annahme, dass ausreichend Material
in der Hochschule vorhanden sei. Infolgedessen überstiegen die tatsächlichen Projektkosten
die ursprünglichen Schätzungen erheblich. Dieses Problem hätte durch eine gründlichere Vor-
planung vermieden werden können. Dennoch ist anzumerken, dass vergleichbare Produkte, die auf dem Markt erhältlich sind, erheblich kostenintensiver sind als das im Rahmen dieses Projekts entwickelte Gerät.

\end{description}